\section{Vision}
\stefan{jeg tror dette afsnit skal væk, det er overlap med \ref{projectvisionrepr}}
Vores vision for projektet repræsenteres ved hjælp af Metaphor, specifikt formuleres det ved hjælp af tre metaforer.
Disse metaforer er \textit{Objektiv dagbog}, \textit{Fitness tracker} og \textit{F16 fly}.
At formulere vores vision som en metafor lader os koncist specificere hvad der er nøgleaspekterne af produktet.

Den objektive dagbog danner tanken om en dagbog baseret på objektive datakilder, hvilket svarer til sensor data, brugsdata etc.
Alt sammen data der kan indsamles uden brugerinteraktion.

Fitness trackeren som metafor planter tanken om en applikation, der løbende evaluerer ens præstationsevne, hvilket kan oversættes til mentalt helbred.

F16 fly metaforen henvender sig til platforms designet, der er tiltænkt at være en meget modulær og kraftig platform, ligesom det er tilfældet med F16 flyet hvor man kan hægte en lang række komponenter på alt efter hvad der er brug for i den pågældende situation.

Andre visions repræsentationstyper der kunne overvejes er \textit{Icon}, \textit{Prototype} og \textit{Proposition}.
\textit{Icon} repræsentationen går på det visuelle, hvor øvelsen er hvordan man konkretiserer en eller flere nøglekvaliteter af en idé.
Dette er eksempelvis hvor den teknologiske del er nemt udførlig, men hvordan det æstetiske i en løsning ikke er umiddelbar.
Vores produkt går mere på indsamling og evaluering af data mere end det æstetiske, hvilket er hvorfor ikon løsningen ikke er valgt.

\textit{Prototype} bruges til at få feedback på en ufuldstændig software løsning og er en fysisk repræsentation af løsningen.
Denne repræsentation kunne med fordel bruges i fremtiden til at få feedback på ens software løsning, men grundet det tidlige stadie i produktet er denne repræsentation ikke benyttet.

\textit{Proposition} er en tekstuel beskrivelse af ens vision. Det svarer basalt set til en traditionel problemformulering.
Det er et alternativ som også kunne være benyttet, men vi finder at den metaforiske tilgang er nemmere at forbinde til end en tekstuel repræsentation i vores tilfælde.
