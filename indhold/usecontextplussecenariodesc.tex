\paragraph{Use context}
Konteksten som applikationen skal kunne bruges i er meget bred, da den omfatter hele patientens hverdag.
Der skal derfor tages højde for at forbindelse til GPS, WIFI og mobilnetværk ikke altid er tilgængelige.
Da applikationen skal logge data om patientens færden skal der sikres håndtering en række forskellige brug/placering af smartphone.
Denne kan eksempelvis være i lommen, i hånden, på et bord eller i lommen på en jakke der hænger i entréen.
Konteksten kan altså variere fra ude i en skov uden dataforbindelse til patientens arbejde.

\subsection{Stakeholders}
I dette projekt er patienterne den vigtigste \textit{stakeholder}, da det er patienterne, der skal bruge systemet i hverdagen.
Systemet skal derfor udvikles på patienternes præmisser.
I dette projekt er det specifikt patienter med unipolar eller bipolar affektiv lidelse vi beskæftiger os med.

Der er et antal sponsorer tilknyttet projektet.
Morten Aagaard har både erfaring inden for datalogi og psykologi og kan agere bindeled mellem de to discipliner.
Janne Vedel Rasmussen og Jørgen Aagaard arbejder inden for psykologifaget og kan derfor bidrage med fagrelevant information.
De har derudover en interesse i at få udviklet værktøjer der kan bidrage til deres arbejde.

\subsection{Scenarios}
Her undersøges hvordan \textit{the Challenge} bliver set fra brugerens perspektiv.
Teknikker til denne undersøgelse inkluderer at udforske systemets problemdomæne ved hjælp af \textit{Use scenarios}.

\paragraph{Use scenarios}
\textit{Use scenarios} bruges til at udforske ideer og muligheder i forhold til brugerens brug af systemet.

\subparagraph{Scenarier:}
\begin{itemize}
	\item Patienten bevæger sig rundt i sin hverdag med smartphonen i lommen. 
	\begin{itemize}
		\item Data logges i systemet om gemmes til senere analyse analysere.
	\end{itemize}
	
	\item Patienten vil gerne have applikationen til at fortælle hvordan den vurderer hans tilstand.
	\begin{itemize}
		\item Applikationen viser at patienten udviser normal adfærd.
		\item Applikationen viser at patientens stemningsleje er lavere end normalt.
		Patienten konsulterer sin liste af lystbetonede aktiviteter og udfører en af disse.
		\item Applikationen viser at patientens stemningsleje er lavere end normalt.
		Patienten foretager sig intet og tilstanden fortsætter med at forværres.
	\end{itemize}
\end{itemize}
