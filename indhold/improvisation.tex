\subsection{Improvisering}
Improvisering er en stor del af software innovation som \citet[side 56]{book:softwareinnovation} siger: \textit{Improvisation and bricolage flesh out the skeleton, whatever the underlying process. Developing technically exploratory software involves manoeuvring in uncharted waters, where development platforms are uncertain and untried, so it is unlikely that generic process models or formal development methods can provide enough support for the developer.}

Derfor har improvisering været en stor del af vores proces, og denne improvisering har taget forskellige former.
Som \citet[side 56]{book:softwareinnovation} siger: \textit{In a development situation, improvisation often takes the form "let's try this...:" a customer meeting, a programming technique, a diagramming technique, a different hardware component.}
Af disse former har vi mødt med interesserede personer og patienter med psykiske lidelse for at få deres perspektiv og deres idéer om platformen og hvad der kan gøres med de data kilder der er tilgængelige. 
Ydermere har vi brugt brainstorming, idet vi så på de forskellige sensorer og kom med idéer om hvad de kunne bruges til for at få kontaktpersonernes holdninger om dette.
På samme tid har vi også brugt diagram teknikker, til at afklare hvordan en arkitektur passende til de krav vi fik gennem kontaktpersoner og patienter skulle se ud.
Teknikker er blevet brugt til at lave flere versioner af arkitekturen og af disse er der blevet foretaget et valg.
