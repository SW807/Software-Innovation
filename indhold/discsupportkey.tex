\section{Discussion of support for key use scenarios}
Ud fra disse scenarier har vi følgende tilstands klassificerings problematikker:
\begin{itemize}
\item På korrekt person?
\item Tilstands ændring?
\item Kritisk tilstand?
\end{itemize}

Den første problematik er hvordan man identificerer hvilken kontekst dataindsamlingsenheden befinder sig i, og ud fra denne vurdere hvilke datakilder der stadig bibeholder deres relevans i denne kontekst.
Eksemplet der et givet i førnævnte scenarie er at patienten har efterladt eller glemt sin telefon i sin jakke, i en sådan situation kunne man forstille sig at datakilder som for eksempel bevægelsesdata eller lydregistrering er ubrugelige.
Dog skal man også i denne sammenhæng være forsigtig, for hvis man forkaster for meget data når telefonen ikke er på en person, vil det gøre det vanskeligt at lave analyser på perioder hvor man ved at telefonen ikke holdes af en person hvilket kunne være søvn analyse. 
Det er dog ikke kun et spørgsmål om at identificere hvornår telefonen ikke er på en person.
Der skal også identificeres om det er den tiltænkte person der bærer enheden, da i disse situationer giver data der indsamles et indblik i den forkerte persons tilstand.

Den anden udfordring, at identificere brugerens tilstand, kræver indsamling af meget data fra mange forskellige kilder for at have et grundlag for en vurdering af disse.
Hvilke datakilder dette skal være kan variere meget fra individ til individ.
Et individ kunne for eksempel have brug for at få set på sit bevægelsesmønster og på sin sociale adfærd, hvor et andet kunne have brug for at få analyseret på søvn og på hvor meget han har forladt sit hjem.
At kunne se på disse er udfordrende i sig selv, da man for hver af disse skal finde ud af hvilke datakilder man har til rådighed der kan give en fornuftig information om disse, og derefter skal man finde ud af hvordan man ud fra den data kan lave en vurdering af det givne kriterie.
Når hvert kriterie er vurderet skal der laves en vurdering af tilstanden som helhed, og hvordan denne laves skal variere fra individ til individ.
Det kunne også være man valgte at inddrage brugeren i denne vurdering, enten ved at give dem spørgsmål eller ved at have dem til at gøre noget andet der kunne give et prej om hvordan de har det.

Den sidste tilstand der kan være er en kritisk tilstand, hvor det vurderes at patienten er så langt nede at der er risiko for han kan forvolde skade på sig selv eller andre.
Her er problemet hvornår man vurderer en tilstand til at være alvorligt kritisk og også hvad man rent faktisk skal gøre hvis man når frem til en sådan vurdering.
Den kritiske vurdering kunne være at man har en grundlinje for hvor lavt eller højt folks sindstilstand kan komme på en scala før den er kritisk, dog er dette nok ikke en god tilgang, da det som altid kan variere meget for hvert individ.
En anden mulighed ville være at se på hvor folk plejer at befinde sig rent tilstandsmæssigt og så ud fra det vurdere hvor langt det er acceptabelt at lade dem svinge fra det.
Den anden del af dette scenarie er der skal gøre hvis det vurderes at en patient er for langt ude i forhold til det habituelle niveau de plejer at have.
Det kunne være det valgtes at der skulle kontaktes en ambulance til enhedens nuværende position, dog ville dette nok kun være tilfældet hvis der er meget stor tillid til estimeringerne fra systemet.
En mere sandsynlig handlingsplan ville være at kontakte læge eller nærmeste pårørende, få dem til at komme med deres vurdering og derefter foretage de handlinger der passer bedst.
Denne mulighed er at foretrække hvis det er muligt for den kontaktede person at komme med en vurdering ud fra information systemet sandsynligvis har givet dem adgang til.