\subsection{Software team dynamik}
Om et software team er innovativt afhænger af en mængde faktorer som teamet sjældent har kontrol over.
Disse faktorer varierer fra omgivelserne de sidder i, til de krav der bliver stillet fra deres overordnede.
Et udvalg af disse faktorer vil i dette blive beskrevet (taget fra \citet[s.~81-82]{book:softwareinnovation}), hvorefter vi vil vurdere deres indflydelse på vores projekt.

\begin{description}[style=nextline]
	\item[Tidspres] Når et team arbejder under tidspres forhindrer det at teamet bruger tid på refleksion, hvilket udelukker at der bruges tid på at udforske og eksperimentere med mulige løsninger.
	\item[Stress] Stress har både fysiologiske og psykologiske effekt på produktiviteten. Når et team er stresset forværres kommunikationen internt hvilket har en negativ effekt på produktet.
	\item[Ressource mangel] Mangel på ressourcer tvinger teamet til de vante arbejdsgange, i stedet for eksperimenterende tilgange til problemer.
	\item[Strengt definerede arbejdsgange] Hvis teamet er tvunget til et sæt af prædefinerede arbejdsgange fra deres overordnede besværliggør det refleksion.
	\item[Bureaukrati] Overdokumentation af teamets arbejde tvinger holdet til at følge de definerede processer 
	\item[Rutinearbejde] For meget rutinearbejde får teamets medlemmer til ikke at se opgaverne i et alternativt perspektiv.
	\item[Dårlig projektstyring] Autoritære projektstyringsstile har en negativ indflydelse på innovative karakteristikker som dialog.
\end{description}

I det følgende vil det blive beskrevet 

\paragraph{Semester}
Da vi er dikterede af studieordningens indhold er der nogle forhold som vi ikke selv er herre over.
For det første løber semestret over en prædefineret periode hvilket introducerer et \textbf{tidspres}.
Studieordningen kræver også at arbejdet bliver dokumenteret i en rapport hvilket er en blanding af \textbf{bureaukrati} og \textbf{strengt definerede arbejdsgange}.

\paragraph{Vejleder}
En barriere der ikke har været i projektet er \textbf{ressource mangel}, da vores vejleder fra starten har bidraget til projektet både med indsigt og med ressourcer som smartphone og smartwatch.

\paragraph{Teamet}
Teamet har i til dels været tilbøjelig til \textbf{rutinearbejde} i forhold til strukturen i projektet.
Tasks defineres, kodes og dokumenteres uden videre eftertanke om, hvorvidt vi er på vej i den rigtige retning.
\textbf{Dårlig projektstyring} er i kilden defineret ud fra autoritære projektstyringsstile.
I vores projekt har det omvendte været problemet, der har ikke været én projektleder, men i stedet har vi alle haft en del af lederrollen.
Dette har betydet at ledelsen har været løs og uden retning.