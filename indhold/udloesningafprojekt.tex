\section{Udløsning af Projektet}
% Essence chapter 4.
Når et projekt startes er det vigtigt at have en fælles forståelse for hvad retning man bevæger sig i.
Denne retning udspringer af det problem som udløste projektet, men kan efterfølgende ændre sig som projektet skrider fremad.

I Essence beskrives 4 typer af initierende problem; \textit{User need}, \textit{Technological opportunity}, \textit{Solution reuse} og \textit{Competitive stress}.
Disse 4 ligger op ad de 4 \textit{views} i Essence; henholdsvis \textit{Paradigm}, \textit{Product}, \textit{Project} og \textit{Process}.

Det der udløste dette projekt var en kombination af de to første; user need og technological opportunity.
Det problem at mange personer har psykiske lidelser har eksisteret i lang tid, så på denne måde er og har der været et brugerbehov, og endda et samfundsbehov, i længere tid.
Problemet ligger både i at detektere og forbygge værre tilfælde.
Yderligere bliver smartphones mere og mere udbredte, samtidig med at der opstår flere og flere krops-knyttede enheder der kan indsamle et væld af informationer om en brugers tilstand og færden.
Deri ligger problemet at udnytte den nye teknologi i forbindelse med det førnævnte problem omkring detektering og forebyggelse af psykiske lidelser.

Kombinationen af disse to initierende problemer har givet anledning til et udgangspunkt til dels i \textit{paradigm} og \textit{product} \textit{views}.
Dette har ført til et behov for en grundig analyse af problemområdet, herunder de forskellige lidelser og hvilke typer af patienter der kunne være interessante for projektet.
Samtidig har der også været fokus på produktet, i det at de mange mulige datakilder der kan tilgås gennem smartphone eller wearables har været undersøgt, samtidig med at der er udviklet en dynamisk platform der kan understøtte adskillige af disse datakilder.
