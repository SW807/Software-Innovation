\section{Example of Idea Maturation}
I dette afsnit vil et eksempel på såkaldt `Idea Maturation' fra \citet[Kapitel 23]{art:essence}.
Specifikt bruger vi en gammel og en ny konfigurationstabel for det implementerede system, og vi undersøger forskellen mellem de to konfigurationer.
Dog er der så mange ændringer, idet den gamle konfigurationstabel blev dannet da systemet var nyt. 
Derfor dækkes kun nogle af ændringer i konfigurationstabellen, idet idéen er at det bare skal være interessant at snakke om og det er derfor ikke nødvendigt at gå i dyb detalje med alting.

De to konfigurationstabeller, der kan ses i \cref{tab:tidligKonfigurationsTabel} og \cref{tab:konfigurationsTabel}, henholdsvis den gamle og den nye. 
I den gamle konfigurationstabel er det blevet markeret hvad der er ændret, men det er kun "Evaluation", "Stakeholders" og "Features" der undersøges. 

Som man kan se i Evaluation, er Procedure blevet ændret fra "fokusgrupper" til "Fokusgruppemøde, Integrationstest" og Criteria er blevet ændret fra "Mortens dogmeregler" til at systemet skal være "Modulært, Fleksibelt, Kombinerbar, og Kommunikativ". 

I "Stakeholders" er de aktuelle stakeholders blevet ændret fra en masse personer til kun at dække "Sponsorer og Patienter", hvor Patienter ses som main perspective og sponsorer er meget mindre vigtige. 
Grunden til at det blev ændret til dette, var at det ikke var helt klart at patienter var de allervigtigste, idet de blev tabt i en liste af forskellige personer.
Ydermere blev Main perspective også ændret lidt, før stod der at det var personer med bipolar affektiv lidelse hvilket blev ændret til unipolar og bipolar affektive lidelser. 

I "Features", blabla

\begin{figure}
	\includegraphics[scale = 0.65,trim = 1cm 3cm 6cm 2cm, angle = 90, clip]{tidlig-konfigurationstabel.pdf}
	\caption{En konfigurationstabel på et tidligt stadie af udviklingen}
	\label{tab:tidligKonfigurationsTabel}
\end{figure}