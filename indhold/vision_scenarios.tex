\section{Vision Scenarios}
Hele dette afsnit er bygger på \citet[Sektion 17.1]{art:essence}, hvor der foreslås en teknik kaldet \textit{vision scenarios} til at nærmere bestemme fokus for et projekt.
For at styre dette projekt i den rigtige retning er der blevet lavet vision scenarios som hjælper med dette.
Til at lave vision scenarios er det nødvendigt at tildele roller til de forskellige personer der er med i projektet.
Rollerne der skal udleves er Child \citep[Kapitel 18]{art:essence}, Challenger \citep[Kapitel 19]{art:essence}, Responder \citep[Kapitel 20]{art:essence} og Anchor \citep[Kapitel 21]{art:essence}.

\subsection{Bestemmelse af retning}
For at hjælpe med at bestemme retning for projektet, hvor der var flere interessante bud, opsættes \textit{vision scenarios}.
Her sættes flere modsætninger overfor hinanden, hvor hver modsætningspar kan føre projektet i to forskellige retninger.
I vores tilfælde fandt vi seks forskellige modsætninger vi syntes kunne være interessante at kigge på:
\begin{itemize}
	\item Én Lidelse kontra Flere lidelser
	\item Personligt kontra Delt med læge
	\item Udregning i skyen kontra Lokal udregning
	\item Opbevaring i skyen kontra Lokal opbevaring
	\item Bruger input kontra Sensor input
	\item Interventioner kontra Patient empowerment
\end{itemize}

Ud af disse seks modsætningspar blev udvalgt de 2 vurderet mest relevante, hvilket blev gjort på demokratisk vis.
Dette førte til en 2-dimensioneal sammenligning, hvor på den ene akse er der opsat \textit{Bruger input} gående mod \textit{Sensor input} og på den anden akse \textit{Interventioner} gående mod \textit{Patient empowerment}.
Derefter anskues alle 4 kombinationer af begreber med de 4 forskellige roller.

\subsection{Child}
Dette er den eneste rolle alle i projektet kan have på vilkårlige tidspunkter.
Det er også denne rolle der sørger for at der kommer kreative idéer og er helt central for hvordan idéer bliver lavet i samarbejde mellem udviklere og kunder.

\subsubsection*{Bruger Input og Interventioner}
Brug bruger input til at opdage påbegyndende forværring i tilstand og oplys patienten om dette.
\subsubsection*{Bruger Input og Patient empowerment}
Brug bruger input til at informere patient om sin nuværende tilstand.
\subsubsection*{Sensor input og Interventioner}
Brug objektive datakilder til at opdage påbegyndende forværring i tilstand og oplys patienten om dette. 
\subsubsection*{Sensor input og Patient empowerment}
Brug objektive datakilder til at informere patienten om sin nuværende tilstand.

\subsection{Challenger}
Det er den der snakker på vegne af \textit{stakeholders}.
Det er også ud fra denne rolle der gives opgaver, prioriteres opgaver og accepteres løsninger.
Derudover skal dem som har rollen være inspirerende i den forstand at de skal kunne få udviklere til at se flere mulige løsninger. 
Challenger bruges på denne måde til at sørge for at den løsning der vælges er den rigtige løsning, samt om det er den bedste løsning der kan blive lavet.

\subsubsection*{Bruger Input og Interventioner}
Ud fra bruger input foreslås lystbetonede aktiviteter.
\subsubsection*{Bruger Input og Patient empowerment}
Visualisere bruger input så brugeren kan blive opmærksom på sine ændringer i stemningsleje.

\subsubsection*{Sensor input og Interventioner} 
Ud fra sensor input registeres ændring i adfærd til at foreslå lystbetonede aktiviteter.

\subsubsection*{Sensor input og Patient empowerment}
Visualisere sensor input så brugeren kan blive opmærksom på sine ændringer i adfærd.

\subsection{Responder}
Responders opgaver består i at lave mulige løsninger om til færdige løsninger og at løse de forskellige opgaver der bliver givet.
Denne rolle består af udviklerne og det er derfor denne rolle der står for prototypes og alle de tekniske løsninger.
Responder rollen bruges til at se om man har udnyttet det fulde potentiale af projektet.
Her er der tale om man bruger de rigtige komponenter, men også om man bruger disse komponenter så meget som man overhovedet kan. 

\subsubsection*{Bruger Input og Interventioner}
Tilpasset dagbog baseret på tidligere dagbogsindlæg.
Struktureret så data kan analyseres og sammenlignes og derudfra foretage interventioner
\subsubsection*{Bruger Input og Patient empowerment}
Elektronisk dagbog med løs tekst.
\subsubsection*{Sensor input og Interventioner} 
Brug sensor data til at foreslå interventioner til brugeren baseret på trends, fx. gang, søvn eller social aktivitet.
\subsubsection*{Sensor input og Patient empowerment}
Visualisere sensor data så patienten kan vurdere sin tilstand.

\subsection{Anchor}
Anchor er den der skal beskytte udviklerne fra at blive forstyrret og glemme hvad fokus er.
Dernæst er det også denne rolle der sørger for at planlægge specielle events, såsom sprint planning.
Anchors rolle er at sørge for at alle stakeholders er tilfredse med det der bliver udviklet.
Derudover er det også deres opgave at beskrive de fordele og ulemper der er ved de forskellige idéer, visioner, prototyper og strategier.

\subsubsection*{Bruger Input og Interventioner}
Hvordan kan det afgøres om man skal foretage interventioner.
Hvordan kan dagbogindslag sammenlignes.
Hvordan vælges hvilken intervention.
\subsubsection*{Bruger Input og Patient empowerment}
Hvordan får brugeren overblik over sin dagbog.
\subsubsection*{Sensor input og Interventioner} 
Hvordan kan det afgøres om der skal foretages en intervention.
Hvordan identificere man treds og ændring i adfærd.
Hvordan vælges hvilken intervention.
\subsubsection*{Sensor input og Patient empowerment}
Hvordan kan man visualisere meget data.
Hvordan kan data aggregeres.
