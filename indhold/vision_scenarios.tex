\section{Vision Scenarios}
Hele dette afsnit er bygger på \citet[Sektion 17.1]{art:essence} der giver indblik i og overblik over hvordan vision scenarios skal bruges.
For at kunne styre projektet i den rigtige retning er der blevet lavet vision scenarios som hjælper med dette.
Til at lave vision scenarios er det nødvendigt at tildele roller til de forskellige personer der er med i projektet.
I vores tilfælde har det ikke været muligt at få andre end os selv med til at lave dette, og vi har derfor prøvet at leve os alle ind i de forskellige roller.
Rollerne der skal leves ind i er Child \citep[Sektion 18]{art:essence}, Challenger \citep[Sektion 19]{art:essence}, Responder \citep[Sektion 20]{art:essence} og Anchor \citep[Sektion 21]{art:essence}.

\begin{description}[style=nextline]
	\item[Child] 
		rollen er den eneste rolle alle i projektet kan have på vilkårlige tidspunkter.
		Derudover er det også denne rolle der sørger for at der kommer kreative idéer.
		Det er også denne rolle der er helt central for hvordan idéer bliver lavet i samarbejde mellem udviklere og kunder. 
	\item[Challenger]
		rollen er den der snakker på vegne af stakeholders.
		Det er også denne rolle der giver projektet opgaver, prioritere opgaver og accepter løsninger.
		Derudover skal dem som har rollen være inspirerende i den forstand at de skal kunne få udviklere til at se flere mulige løsninger. 
	\item[Responder]
		Rollen består i at lave mulige løsninger om til færdige løsninger, og er dem der sørger for at løse de forskellige opgaver der bliver givet.
		Denne rolle består af udviklerne, og det er derfor denne rolle der står for prototypes og alle de tekniske løsninger.
	\item[Anchor]
		Rollen er den der skal beskytte udviklerne fra at blive forstyrret og glemme deres fokus.
		Dernæst er det også denne rolle der sørger for at planlægge specielle events, såsom sprint planning.
\end{description}

For at finde ud af hvad der skulle være på de forskellige akser blev der givet eksempler på hvad de forskellige akser kunne indeholde.
I vores tilfælde fandt vi seks forskellige ting vi synes kunne være interessante at kigge på, hvilket er som følger.
\begin{itemize}
	\item En Lidelse mod Flere lidelser
	\item Personligt mod Delt med læge
	\item Udregning i skyen mod Lokalt udregning
	\item Opbevaring i skyen mod Lokal opbevaring
	\item Bruger input mod Sensor input
	\item Interventioner mod Patient empowerment
\end{itemize}

For at kunne vælge hvad for nogle vi mente var mest interessante kigge på sammenlignede vi de forskellige akser mod hinanden i forhold til hvad vi synes var mest interessant.
Resultatet af dette er at bruger input mod sensor input skulle være på den ene akse og interventioner mod patient empowerment på den anden akse.
Herefter er der så prøvet hvor vi lever os ind i de forskellige roller for at se hvordan de ser projektet udvikle sig i forhold til hvilken retning der er snak om.

\subsection{Child}
Childs rollen er en rolle der beskriver hvor projektet kan gå hen hvis man ser simpelt på de forskellige akser.

\subsubsection*{Bruger Input og Interventioner}
Brug bruger input til at intervenere patienter.
\subsubsection*{Bruger Input og Patient empowerment}
Brug bruger input til at informere patient nuværende situation.
\subsubsection*{Sensor input og Interventioner}
Brug objektive datakilder til at intervenere patienter. 
\subsubsection*{Sensor input og Patient empowerment}
Brug objektive datakilder til at informere patient nuværende situation.

\subsection{Challenger}
Challenger rollen bruges til at sørge for at den løsning der vælges er den rigtige løsning, samt om det er den bedste løsning der kan blive lavet.

\subsubsection*{Bruger Input og Interventioner}
Ud fra bruger input forslå lystbetonede aktiviteter.
\subsubsection*{Bruger Input og Patient empowerment}
Visualisere bruger input så brugeren kan blive opmærksom på sin stemninglejehistorik.
\subsubsection*{Sensor input og Interventioner} 
Ud fra sensor input registeres ændring i adfærd til at foreslå lystbetonede aktiviteter.
\subsubsection*{Sensor input og Patient empowerment}
Visualisere sensor input så brugeren kan blive opmærksom på sin adfærdshistorik.

\subsection{Responder}
Responder rollen bruges til at se om man har udnyttet det fulde potentiale af projektet.
Her er der tale om man bruger de rigtige komponenter, men også om man bruger disse komponenter så meget som man overhovedet kan. 

\subsubsection*{Bruger Input og Interventioner}
Tilpasset dagbog baseret på tidligere dagbogsindlæg.
Struktureret så data kan analyseres og sammenlignes og derudfra foretage interventioner
\subsubsection*{Bruger Input og Patient empowerment}
Elektronisk dagbog med løs tekst.
\subsubsection*{Sensor input og Interventioner} 
Brug sensor data til at foreslå interventioner til brugeren baseret på trends, fx. gang, søvn eller social aktivitet.
\subsubsection*{Sensor input og Patient empowerment}
Visualisere sensor data så patienten kan vurdere sin tilstand.

\subsection{Anchor}
Anchors rolle er at sørge for at alle stakeholders er tilfredse med det der bliver udviklet.
Derudover er det også deres opgave at beskrive de fordele og ulemper der er ved de forskellige idéer, visioner, prototyper og strategier.

\subsubsection*{Bruger Input og Interventioner}
Hvordan kan det afgøres om man skal foretage interventioner.
Hvordan kan dagbogindslag sammenlignes.
Hvordan vælges hvilken intervention.
\subsubsection*{Bruger Input og Patient empowerment}
Hvordan får brugeren overblik over sin dagbog.
\subsubsection*{Sensor input og Interventioner} 
Hvordan kan det afgøres om der skal foretages en intervention.
Hvordan identificere man treds og ændring i adfærd.
Hvordan vælges hvilken intervention.
\subsubsection*{Sensor input og Patient empowerment}
Hvordan kan man visualisere meget data.
Hvordan kan data aggregeres.

\subsection{Resultat}






