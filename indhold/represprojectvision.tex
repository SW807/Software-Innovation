\section{Representation of Project Vision}
Der findes flere måder at præsentere sin vision på. 
I \citet[Kapitel 24 - Representation]{art:essence} præsenteres fire repræsentationer, \textit{Metafor}, \textit{Ikon}, \textit{Prototype} og \textit{Proposition}.

Vi har valgt at repræsentere vores vision ved hjælp af metaforer, da denne repræsentation er abstrakt og giver meget plads til fortolkning.
Metafor-repræsentationen giver en beskrivelse af hvad fokusområdet er, uden at afgrænse fra at se på andre retninger.

De tre metaforer vi har benyttet er \textit{Objektiv dagbog}, \textit{Fitness tracker} og \textit{F-16 fly}.

\textbf{Den objektive dagbog} danner tanken om en dagbog baseret på objektive datakilder, hvilket svarer til sensor data, brugsdata etc.
Alt sammen data, der kan indsamles uden direkte brugerinteraktion, altså uden at brugeren bevidst gør ting der har effekt på den indsamlet data.

\textbf{Fitness trackeren} som metafor planter tanken om en applikation der løbende evaluerer ens præstationsevne, hvilket kan oversættes til helbred, herunder mentalt helbred.

\textbf{F-16 fly}\label{vision::fly} metaforen henvender sig til designet af platformen, der er tiltænkt at være modulær, ligesom det er tilfældet med F-16 flyet, hvor man kan hægte en lang række komponenter på alt efter hvad der er brug for i den pågældende situation.
Her har vi varierende symptomer, hvilket kræver at indsamling og analyse af data kan skifte efter behov.
