\section{Kriterier}\label{firstsubseckriterier}
Her bliver de vigtigste kriterier præsenteret, som skal opfyldes for at kunne kalde projektet en succes.
Disse kriterier er de samme som kan findes i konfigurationstabellen (\cref{tab:konfigurationsTabel}).
Kriterierne opstilles til at kunne evaluere ens løsning.

\begin{description}[style=nextline]
	\item[Modulær] 
	Da individer kan have forskellige symptomer skal det være muligt at tilføje og fjerne datakilder, samt analyser af disse, så den enkelte patient får den bedst mulige behandling.
	\item[Fleksibel]
	Det skal være nemt at modificere funktionalitet til platformen, da platformen skal kunne tilpasses til individet.
	\item[Kombinerbar] Eftersom vi prioriterer modulærbarhed skal ansvarsområder være lette at separere så indhentede data kan bruges på tværs af systemet.
	\item[Kommunikativ] Data skal være tilgængeligt på tværs af systemet, men skal samtidig være beskyttet mod at blive redigeret af uvedkommende.
\end{description}

Kriterierne opstilles til at evaluere ens løsning.
Resultaterne af denne evaluering  noteres i ``findings'' og danner grundlag for den næste konfigurationstabel.
Der kan læses om et eksempel på brug af kriterier i \citet[Kapitel 2.2, 2.3, 2.4 og 2.5 side 16--21]{art:essence}.

\subsection{Evaluering af kriterier}
\begin{description}[style=nextline]
	\item[Modulær]
	Hele arkitekturen er opbygget af en manager applikation og en række separate moduler, som kan bruges i manageren.
	Dette understøttes af den fælles datagrænseflade i form af DBAccess og moduldefinitionerne.  
	Det er muligt at specificere at man er afhængig af blot et modul eller af en mængde af disse.
	Eksempelvis er det muligt for et analysemodul at afhænge af et accelerationsmodul.
	
	Som et eksisterende eksempel på at moduler nemt kan tilføjes og bruges af manageren er udviklede moduler, beskrevet i \citet{misc:soevnrapp} og \citet{misc:surveyrapp}.
	Disse eksempler viser også hvorledes det er muligt at udvikle moduler, der afhænger af data fra andre moduler.
	
	\item[Fleksibilitet]
	Til at sikre at platformen er fleksibel i forhold til patienten, hvor de har kontrol over hvilke moduler, der kan køre, tilbydes dette i form af en indstillingsmenu.
	Endvidere har patienten også kontrol over hvilke moduler, der er installeret på deres smartphone, idet hvert modul er en separat applikation på deres smartphone, som de sagtens kan afinstallere hvis ønsket.
	
	\item[Kombinerbar]
	Data indsamlet fra diverse moduler kan benyttes af andre moduler, eksempelvis analysemoduler.
	Et klart eksempel på dette kan læses i \citet{misc:soevnrapp}, hvor data fra et søvnestimeringsmodul for acceleration og et modul for amplitude kombineres i et samlet søvnestimeringsmodul.
	
	\item[Kommunikativ]
	Det kommunikative kriterie er understøttet i den grad at data nemt kan kommunikeres til relevante moduler.
	Dog er sikkerhedskriteriet for denne kommunikation \textit{ikke} på plads.
	Derudover er der ikke opsat skriverettighedsrestriktioner, og således er der altså et kriterie der ikke er blevet opfyldt og skal arbejdes videre med før man kan konkludere at en tilpas færdig platform er udviklet.
\end{description}

Ud fra denne evaluering har vi en finding der gå på manglende skriverettighedsrestriktioner.
Desuden kan vi erfare at kriterierne om modularitet, fleksibilitet og kombinerbarhed er opfyldt.
