\subsection{Community}
% Udvikling sker ikke i vakum
% noget om social interaktion og ny viden
Som nævnt i \citet[Kapitel 2]{book:softwareinnovation} foregår software-udvikling sjældent i et vakuum.
Den meste udvikling foregår som del af en social proces, dette er især gældende for innovativ udvikling.
Ved at arbejde indenfor communities, enten åbne eller lukkede, er der større mulighed for at lufte og udforske idéer og derved få afprøvet og bekræftet disse, samt mulighed for videre udvikling med andre interesserede.
I disse communities er det også muligt at finde eksperter indenfor andre områder, så der også på denne måde kan deles allerede eksisterende viden, som yderlige kan assistere ved tilblivelsen af nye idéer.

Disse ting gør sig også gældende for vores projekt, hvor vi er interesseret i at både dele og opnå viden, for at kunne levere det bedst mulige produkt, altså hvor vi bedst muligt kan hjælpe personer med psykiske lidelser.
Dette er viden fra andre udviklere, fagfolk og potentielle brugere.

\subsubsection{En åben platform}
% Åben/fleksibel struktur/arkitektur
% Mulighed for udvikling af nye moduler og deling af disse
% Udviklere vha kode, fagfolk vha grafisk værktøj. lissom Mindstorms
PsyLog platformen er udviklet som en fleksibel struktur, der tillader uafhængig udvikling af nye moduler.
Strukturen anvender moduler som byggesten ved, for det første, at lade patienten bestemme kombinationen af moduler på hans mobiltelefon og for det andet ved at tillade at alle kan udvikle nye moduler.

Herved kan systemet betragtes som et fælles system som alle har mulighed for at bidrage til.
Platformen og de moduler der er udviklet hertil er tilgængelige via \href{http://github.com}{github.com}.
På samme måde kan moduler udviklet i fremtiden deles via en open source platform.
Med denne tilgang kan man forestille sig en situation hvor innovation kan ske parallelt på flere uafhængige lokationer.
Altså kan man forestille sig en situation hvor man fx ved psykiatrien i Aalborg (eventuelt i forbindelse med semesterprojekter ved Aalborg universitet) udvikler moduler med særligt fokus på social aktivitet, mens man i København fokuserer på søvnbesvær.
På den måde kan udviklingen af nye og forbedringen af eksisterende moduler ske de steder hvor fokus på dem er særligt store.
Dermed kan den faglige ekspertise fra psykiatere og psykologer inddrages til udvikling af moduler der særligt relaterer sig til deres fokusområde.

\subsubsection{Udviklingsmetoder}
I kraft af at udviklingen af PsyLog platformen er sket i forbindelse med et semesterprojekt på en software uddannelse er kravene for udviklingen af nye moduler tilsvarende tekniske.
For at løse denne problemstilling kunne en løsning, som den LEGO har anvendt til deres Mindstorms produkter, anvendes.
Her programmeres en robot ved hjælp af et grafisk \textit{''programmeringssprog''} for derved at gøre udviklingen mere tilgængelig.

For PsyLog kunne der anvendes en lignende tilgang hvor fagfolk (psykiatere og psykologer) kombinerer eksisterende moduler og et mængde simple operationer for derved at skabe nyt indhold.