\subsection{Community}
% Udvikling sker ikke i vakum
% noget om social interaktion og ny viden
Som nævnt i \citet[Kapitel 2]{book:softwareinnovation} foregår software-udvikling sjældent i et vakuum.
Den meste udvikling foregår som del af en social proces, dette er især gældende for innovativ udvikling.
Ved at arbejde indenfor communities, enten åbnede eller lukkede, er der større mulighed for at lufte og udforske idéer og derved få afprøvet og bekræftet disse, samt mulighed for videre udvikling med andre interesserede.
I disse communities er det også muligt at finde eksperter indenfor andre områder, så der også på denne måde kan deles allerede eksisterende viden, som yderlige kan assistere ved tilblivelsen af nye idéer.

Disse ting gør sig også gældende for vores projekt, hvor vi er interesseret i at både dele og opnå viden, for at kunne levere det bedst mulige produkt, altså hvor vi bedst muligt kan hjælpe personer med psykiske lidelser.
Dette er viden fra andre udviklere, fagfolk og potentielle brugere.

% Åben/fleksibel struktur/arkitektur
% Mulighed for udvikling af nye moduler og deling af disse
% Udviklere vha kode, fagfolk vha grafisk værktøj. lissom Mindstorms
