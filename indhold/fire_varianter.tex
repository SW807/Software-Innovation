\section{Relationen mellem projekt og de fire varianter}
%A relation of your project to software innovation variants (see Essence-book Section 5.1).
Dette afsnit beskriver mulige syn på software innovation i lyset af de fire forskellige varianter heraf\citet[Afsnit 5.1, Side 30-31]{art:essence}.
Afsnittet beskriver herved fire tilgange til software innovation i kontekst af PsyLog projektet.

\subsection{Produkt innovation}
% Product innovation is about developing new or improved software-intensive products.

\subsection{Proces innovation}
% Process Innovation is about developing software-intensive solutions that offer users new or improved ways to produce their existing products or services.
Indsamling af information om personlig adfærd hos en patient foretages pt ved at patienten løbende tager noter om adfærd eller ved at foretage interviews.
Noterne kan anvendes til personlig refleksion mens interviews vil give en psykolog eller psykiater mulighed for at klarlægge patientens normale adfærd.
Med en af disse typer information bliver det muligt at registrere ændringer i adfærd.
Dog er begge typer information påvirket af patientens subjektive vurdering.

Hertil kan et system som PsyLog anvendes til at indsamle information om patientens adfærd.
Herved opnås to forbedringer på ovenstående proces.
For det første vil information indsamles kontinuert og uafhængigt af kontekst.
Altså indsamles information ikke kun i situationer hvor patienten ønsker at notere dette eller i forbindelse med forudplanlagte interviews.
Derudover fjernes brugerens subjektive påvirkning af informationer og der skabes derved et mere objektivt billede af patientens egentlige adfærd.

\subsection{Projekt innovation}
% Project innovation is when existing software solutions are fitted into new application domains (illustrated in Section 4.3) – what Tidd et al. labels position innovation.

\subsection{Paradigme innovation}
% Finally, Paradigm innovation is when software-intensive solutions are coupled with changes in the ‘mental models’ which frame what an organization is about; who the users are; or what the market is or wants (illustrated in Section 4.1).
